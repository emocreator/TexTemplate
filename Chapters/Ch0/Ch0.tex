%%%%%%%%%%%%%%%%%%%%%%%% PREAMBLE %%%%%%%%%%%%%%%%%%%%%%%%
\documentclass[../../Root.tex]{subfiles}


\begin{document}
%%%%%%%%%%%%%%%%%%%%%%% DOCUMENT %%%%%%%%%%%%%%%%%%%%%%%

\section{Introduction}
\label{sec:Intro}
\dfn[set:def]{Set}{
    A set is a collection of distinct objects, considered as an object in its own right.
    \[
    \{a,b,c\}=\{b,c,a\}  
    \] 
}
\nota[]{Test}{
tt
}
As defined in \cref{set:def}, a set can contain any type of objects.

\thm[unique-element]{Unique Element Theorem}{
    Every non-empty set has at least one element.
}
Refer to \cref{unique-element} for more details.
\section{Basic Concepts}
\dfn[empty-set]{Empty Set}{
    An empty set is a set with no elements.
}
As defined in \cref{empty-set}, the empty set plays a fundamental role in set theory.
\thm[de-morgan]{De Morgan's Laws}{
    The complement of the union of two sets is the intersection of their complements, and vice versa.
}
\section{Further Discussions}
\dfn[infinite-set]{Infinite Set}{
    An infinite set is a set that is not finite; it has no last element.
}
As defined in \cref{infinite-set}, infinite sets are crucial in various mathematical contexts.
\thm[countable-infinite]{Countable Infinite Sets}{
    A set is countably infinite if its elements can be put into a one-to-one correspondence with the natural numbers.
}

$\int f(x) dx$

$\int_x^a f(x) dx$
\nt{
He
}

\prop{}{HI}

\lemma{}{hi}

Refer to \cref{countable-infinite} for more insights.
\end{document}
